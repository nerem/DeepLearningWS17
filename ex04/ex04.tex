\documentclass[a4paper, reqno]{amsart}
\usepackage{amssymb,amsmath,amsfonts,amsthm,mathrsfs}
%\usepackage{showkeys}
\usepackage[colorlinks=true, citecolor=blue, anchorcolor=red]{hyperref}
\usepackage{enumerate}
\usepackage{underscore}

\newcommand{\mc}[1]{\mathcal #1}
\newcommand{\supp}{\operatorname{supp}}
\newcommand{\R}{\mathbb R}
\newcommand{\N}{\mathbb N}
\newcommand{\C}{\mathbb C}
\newcommand{\Z}{\mathbb Z}
\newcommand{\E}{\mathbb E}
\newcommand{\F}{\mathbb F}
\newcommand{\G}{\mathbb G}
\renewcommand{\Re}{\mathop{\text{\upshape{Re}}}}
\renewcommand{\Im}{\mathop{\text{\upshape{Im}}}}
\newcommand{\cl}{\text{\upshape{cl}}}

\newcommand{\op}{\mathop{\text{\upshape{op}}}}
\newcommand{\rank}{\mathop{\text{\upshape{rank}}}}
\newcommand{\ms}[1]{\mathscr{#1}}
\newcommand{\scp}[2]{\langle #1,#2\rangle}
\newcommand{\cal}[1]{{\mathcal #1}}
\renewcommand{\Im}{\operatorname{Im}}
\newcommand{\ord}{\operatorname{ord}}
\newcommand{\diag}{\operatorname{diag}}
\newcommand{\const}{\operatorname{const}}
\renewcommand{\epsilon}{\varepsilon}
\renewcommand{\bar}[1]{\overline{#1}}
\newcommand{\id}{\mathop{\textrm{id}}\nolimits}
\newcommand{\norm}[1]{\left\lVert#1\right\rVert}
\renewcommand{\mod}[1]{\left|#1\right|}
\renewcommand{\tilde}{\widetilde}
\renewcommand{\phi}{\varphi}

\newtheorem{theorem}{Theorem}[section]
\newtheorem{lemma}[theorem]{Lemma}
\newtheorem{corollary}[theorem]{Corollary}
\newtheorem{proposition}[theorem]{Proposition}
\theoremstyle{definition}
\newtheorem{definition}[theorem]{Definition}
\newtheorem{remark}[theorem]{Remark}

\numberwithin{equation}{section}
\renewcommand{\theequation}{\arabic{section}-\arabic{equation}}

\begin{document}



\title[Exercise 4]
{Deep Learning in Computer Vision - Exercise 4}
\author{Karsten Herth}
\author{Felix Hummel}
\author{Felix Kammerlander}
\author{David Palosch}
\thanks{}
\date{January 23, 2018}

\maketitle

{\bf Exercise 4.1c} \\	The \texttt{tf.nn.conv2d_transpose()} requires to define the output shape since it is not necessarily unique. For example, let the input shape for a convolution be
	\begin{enumerate}
	[(a)]
		\item $4 \times 4,$
		\item $3 \times 3.$
	\end{enumerate}
	In both cases we use a filter of size $3 \times 3$, striding of $2$ and the padding option SAME. Then, in both cases (a) and (b), performing a \texttt{conv2d()} yields
	an output of $2 \times 2.$ Hence, if we want to go back using \texttt{tf.nn.conv2d_transpose()}, we have to tell tensorflow if we came from the $4 \times 4$ input or the
	$3 \times 3$ input. \\
	The \texttt{tf.layers.conv2d_transpose()} uses the $4 \times 4$ output by default. See also our file ex_04_1_c.ipynb, where we implemented this example.

\end{document}