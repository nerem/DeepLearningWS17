\documentclass[a4paper, reqno]{amsart}
\usepackage{amssymb,amsmath,amsfonts,amsthm,mathrsfs}
%\usepackage{showkeys}
\usepackage[colorlinks=true, citecolor=blue, anchorcolor=red]{hyperref}
\usepackage{enumerate}

\newcommand{\mc}[1]{\mathcal #1}
\newcommand{\supp}{\operatorname{supp}}
\newcommand{\R}{\mathbb R}
\newcommand{\N}{\mathbb N}
\newcommand{\C}{\mathbb C}
\newcommand{\Z}{\mathbb Z}
\newcommand{\E}{\mathbb E}
\newcommand{\F}{\mathbb F}
\newcommand{\G}{\mathbb G}
\renewcommand{\Re}{\mathop{\text{\upshape{Re}}}}
\renewcommand{\Im}{\mathop{\text{\upshape{Im}}}}
\newcommand{\cl}{\text{\upshape{cl}}}

\newcommand{\op}{\mathop{\text{\upshape{op}}}}
\newcommand{\rank}{\mathop{\text{\upshape{rank}}}}
\newcommand{\ms}[1]{\mathscr{#1}}
\newcommand{\scp}[2]{\langle #1,#2\rangle}
\newcommand{\cal}[1]{{\mathcal #1}}
\renewcommand{\Im}{\operatorname{Im}}
\newcommand{\ord}{\operatorname{ord}}
\newcommand{\diag}{\operatorname{diag}}
\newcommand{\const}{\operatorname{const}}
\renewcommand{\epsilon}{\varepsilon}
\renewcommand{\bar}[1]{\overline{#1}}
\newcommand{\id}{\mathop{\textrm{id}}\nolimits}
\newcommand{\norm}[1]{\left\lVert#1\right\rVert}
\renewcommand{\mod}[1]{\left|#1\right|}
\renewcommand{\tilde}{\widetilde}
\renewcommand{\phi}{\varphi}

\newtheorem{theorem}{Theorem}[section]
\newtheorem{lemma}[theorem]{Lemma}
\newtheorem{corollary}[theorem]{Corollary}
\newtheorem{proposition}[theorem]{Proposition}
\theoremstyle{definition}
\newtheorem{definition}[theorem]{Definition}
\newtheorem{remark}[theorem]{Remark}

\numberwithin{equation}{section}
\renewcommand{\theequation}{\arabic{section}-\arabic{equation}}

\begin{document}



\title[Exercise 1]
{Deep Learning in Computer Vision - Exercise 3}
\author{Karsten Herth}
\author{Felix Hummel}
\author{Felix Kammerlander}
\author{David Palosch}
\thanks{}
\date{January 7, 2018}

\maketitle

{\bf Exercise 3.1} \\
Let $N, M \in \N.$ Further, let $f\colon \R^n \to \R^m$ and $g\colon \R^m \times \R^m \to \R$ both be differentiable. Let $h \colon \R^n \to \R$ be defined by
$h(x) := g(f(x), f(x)).$ \\
Then, $h$ is differentiable with
\begin{align*}
	(Dh)(x) = (Dg)(f(x), f(x)) \cdot \begin{pmatrix} (Df)(x) \\ (Df)(x) \end{pmatrix} \in \R^{1 \times N}
\end{align*}
for all $x \in \R^N.$

\begin{proof} \hfill \\
	For any $x \in \R^N$ the matrix $(Df)(x)$ is of size $M \times N$
	and for all $x \in \R^{2M} \cong \R^M \times \R^M$ the matrix $(Dg)(x)$ is of size $1 \times 2M.$\\
	We define the functions
	\begin{align*}
		h_1(x) & := (f(x), 0) \in \R^{1\times 2M} \\
		h_2(x) & := (0, f(x)) \in \R^{1 \times 2M} \\
		h_3(x) & := h_1(x) + h_2(x) \in \R^{1\times 2M}
	\end{align*}
	for $x \in \R^N$, such that $h(x) = g(h_3(x)).$
	Obviously, $h_1$ and $h_2$ (and therefore, $h_3$ as well) are differentiable with derivatives
	\begin{align*}
		(Dh_1)(x) & = \begin{pmatrix}
			\nabla f_1(x)^\top \\
			\vdots \\
			\nabla f_M^\top(x) \\
			0 \\
			\vdots \\ 
			0
		\end{pmatrix}
		= \begin{pmatrix}
		 (Df)(x) \\ 0
		\end{pmatrix}
		\in \R^{2M \times N} \\
		(Dh_2)(x) & = \begin{pmatrix}
			0 \\
			\vdots \\
			0 \\
			\nabla f_1(x)^\top \\
			\vdots \\
			\nabla f_M(x)^\top
		\end{pmatrix}
		= \begin{pmatrix}
		 0 \\ (Df)(x)
		\end{pmatrix}
		\in \R^{2M \times N}
	\end{align*}
	for $x \in \R^N.$ \\
	In particular, $h = g \circ h_3$ is differentiable by the chain rule and we obtain, using the linearity of the derivative,
	\begin{align*}
		(Dh)(x)
			& = (Dg)(h_3(x)) \cdot (Dh_3)(x) \\
			& = (Dg)(f(x), f(x)) \cdot \left[ (Dh_1)(x) + (Dh_2)(x) \right] \\
			& = (Dg)(f(x), f(x)) \cdot \begin{pmatrix} 	\nabla f_1(x)^\top \\ \vdots \\ \nabla f_M(x)^\top \\ \nabla f_1(x)^\top \\ \vdots \\ \nabla f_M(x)^\top \end{pmatrix} \\
			& = (Dg)(f(x), f(x)) \cdot \begin{pmatrix} (Df)(x) \\ (Df)(x) \end{pmatrix} \in \R^{1\times N}
	\end{align*}
	for $x \in \R^N.$
\end{proof}

\end{document}